\documentclass[titlepage,11pt,twoside]{article}



\usepackage[dvips]{graphicx}

\usepackage[myheadings]{fullpage}
\usepackage{pmetrika}
\usepackage{pmbib}


\usepackage{times}
\usepackage{epsfig}
\usepackage{graphicx}
\usepackage{amsmath}
\usepackage{amssymb}
\usepackage{subfigure}
\usepackage{upgreek}
\usepackage{multirow}
\usepackage{color}
\usepackage{bm}
\DeclareMathOperator*{\argmin}{arg\,min}
\usepackage{arydshln}
\usepackage{cite}

%\usepackage{submit}

\newcommand{\bfU}{\mbox{\boldmath$\mathsf{U}$}}
\newcommand{\bfu}{\mbox{\boldmath$\mathsf{u}$}}

\newcommand{\Eta}{\mbox{$\mathsf{H}$}}
\newcommand{\subEta}{\mbox{\scriptsize $\mathsf{H}$}}
\newcommand{\uni}{\mbox{\scriptsize $\mathsf{UNI}$}}

%\raggedbottom
\flushbottom


%\firstpage{1}
%\setcounter{lastpage}{999}
\setcounter{secnumdepth}{3}

\begin{document}


\linespacing{1}

\title{Solutions of Orthogonal Procrustes Problems under Partially Known  Prior}

\author{Jun Xu }

\markboth{Psychometrika}{ }

\affil{Department of Computing, The Hong Kong Polytechnic University}


\linespacing{1}

%\RepeatTitle{Psychometrics: From Practice to Theory and Back}

\begin{center}\vskip3pt


\vspace{32pt}

Abstract\vskip3pt

\end{center}


\begin{abstract}
The orthogonal Procrustes problem aims to find an orthogonal matrix which transforms one given matrix into another by minimizing their Frobenius matrix norm. This problem can be applied in applications such as permutation theory, machine learning, and camera calibration, \emph{etc}. In real cases, the permutation matrix may have been partially known, the dictionaries can be partially learned from external data, and the calibration of camera should be done under some fixed priors. This prior information makes the original orthogonal Procrustes problem more difficult. In this paper, we consider the solution of this problem under partially known priors, which includes the original orthogonal Procrustes problem as a special case with no such prior.
\begin{keywords}
orthogonal Procrustes problem, pratially known priors
\end{keywords}
\end{abstract}

\vspace{\fill}\newpage

\section{Introduction}
The classical orthogonal Procrustes problem has been applied in psychometrics, multidimensional scaling, factor analysis, machine learning, computer vision, optical imaging, and robotics.

\section{Definition of the Probelm and Solution}
Let $\mathbf{A},\mathbf{B}\in \mathcal{R}^{n\times m}$ be two given data matrices. Define $\mathbf{X}\in\mathcal{R}^{n\times p}$ and $\mathbf{P}\in\mathcal{R}^{n\times q}$ where $p+q=n$. $\mathbf{X}$ is the partially known prior which could be used to guide the solutions of $\mathbf{P}$. We formulate the orthogonal Procrustes problem with partially known priors as:
\begin{equation}
\mathbf{\hat{P}}=\arg\min_{\mathbf{D}}\|\mathbf{B}-[\mathbf{X}\ \mathbf{P}]\mathbf{A}\|_{F}^{2}
\quad
s.t.
\quad
\mathbf{P}^{\top}\mathbf{P} = \mathbf{I}_{q\times q}, \mathbf{X}^{\top}\mathbf{P} = \mathbf{0}_{p\times q}.
\end{equation} 
In fact, as have been proofed, if the matrix $\mathbf{\mathbf{B}\mathbf{A}^{\top}}$ has no zero singular value, then the solution of  
$\mathbf{\hat{P}} = \mathbf{U}\mathbf{V}^{\top}$ is unique and we do not need any preceding results.

We crop the matrix $\mathbf{A}$ into two parts: $\mathbf{A}_{X}\in\mathcal{R}^{p\times m}$ and $\mathbf{A}_{P}\in\mathcal{R}^{q\times m}$ to interact with $\mathbf{X}$ and $\mathbf{P}$, respectively. Then we have 
\begin{equation}
\begin{split}
&
\|\mathbf{B}-[\mathbf{X}\ \mathbf{P}]\mathbf{A}\|_{F}^{2}
=\|\mathbf{B}-[\mathbf{X}\ \mathbf{P}][\mathbf{A}_{X}^{\top}\ \mathbf{A}_{P}^{\top}]^{\top}\|_{F}^{2}
=\|\mathbf{B}-[\mathbf{X}\ \mathbf{P}][\mathbf{A}_{X}^{\top}\ \mathbf{A}_{P}^{\top}]^{\top}\|_{F}^{2}
\\
&
=\|\mathbf{B}-\mathbf{X}\mathbf{A}_{X}^{\top} - \mathbf{P}\mathbf{A}_{P}^{\top}]\|_{F}^{2}
=\|\mathbf{B}-\mathbf{X}\mathbf{A}_{X}^{\top} - \mathbf{P}\mathbf{A}_{P}^{\top}]\|_{F}^{2}
\end{split}
\end{equation}
The $\mathbf{B}-\mathbf{X}\mathbf{A}_{X}^{\top}$ is a known data matrix and we replace it with 
$\mathbf{B}^{*}=\mathbf{B}-\mathbf{X}\mathbf{A}_{X}^{\top}$. In the following Results 1, we remove the notation $*$ and use $\mathbf{B}$ as the finally known data matrix.

\textbf{Results 1}: Let $\mathbf{A}\in \mathcal{R}^{q\times m}$, $\mathbf{B}\in \mathcal{R}^{n\times m}$ be two given data matrices, given partially known prior of $\mathbf{X}^{\top}\mathbf{X}=\mathbf{I}_{p\times p}$. Then the sufficiency and necessary conditions of
\begin{equation}
\mathbf{\hat{P}}=\arg\min_{\mathbf{P}}\|\mathbf{B}-\mathbf{P}\mathbf{A}\|_{F}^{2}
\quad
s.t.
\quad
\mathbf{P}^{\top}\mathbf{P} = \mathbf{I}_{q\times q}, \mathbf{X}^{\top}\mathbf{P} = \mathbf{0}_{p\times q} 
\end{equation}
is $\mathbf{\hat{P}} = \mathbf{U}\mathbf{V}^{\top}$, where $\mathbf{U}$ and $\mathbf{V}$ are the orthogonal matrices obtained by perfroming economy (aka. reduced) SVD:
\begin{equation}
(\mathbf{I}_{n\times n}-\mathbf{X}\mathbf{X}^{\top})\mathbf{B}\mathbf{A}^{\top} = \mathbf{U}\mathbf{\Sigma}\mathbf{V}^{\top}
\end{equation}

\emph{proof}: 
Since $\mathbf{P}^{\top}\mathbf{P} = \mathbf{I}_{q\times q}$, we have
\begin{equation}
\begin{split}
\mathbf{\hat{P}}
&
=\arg\min_{\mathbf{P}}\|\mathbf{B}-\mathbf{P}\mathbf{A}\|_{F}^{2}
=\arg\min_{\mathbf{P}}\|\mathbf{B}\|_{F}^{2}+\|\mathbf{P}\mathbf{A}\|_{F}^{2}-2\text{Tr}(\mathbf{B}^{\top}\mathbf{P}\mathbf{A})
=\arg\max_{\mathbf{P}}\text{Tr}(\mathbf{A}\mathbf{B}^{\top}\mathbf{P}).
\end{split}
\end{equation}
We can use Lagrange multiplier method and define the Lagrange function as:
\begin{equation}
\mathcal{L}
=
\text{Tr}(\mathbf{A}\mathbf{B}^{\top}\mathbf{P})
-
\text{Tr}(\Gamma_{1}(\mathbf{P}^{\top}\mathbf{P} - \mathbf{I}_{q\times q}))
-
\text{Tr}(\Gamma_{2}(\mathbf{P}^{\top}\mathbf{X}))
,
\end{equation}
where $\Gamma$ is the augmented Lagrange multiplier. Take the derivative of $\mathcal{L}$ with respect to $\mathbf{P}$ and set it to 0, we can get
\begin{equation}
\frac{\partial \mathcal{L}}{\partial \mathbf{P}} 
=
\mathbf{B}\mathbf{A}^{\top}
-
\mathbf{P}(\Gamma_{1}+\Gamma_{1}^{\top})
-
\mathbf{X}\Gamma_{2}^{\top}
\end{equation}
After setting the equation (7) to zero, we get that
\begin{equation}
\mathbf{B}\mathbf{A}^{\top}
-
\mathbf{P}(\Gamma_{1}+\Gamma_{1}^{\top})
-
\mathbf{X}\Gamma_{2}^{\top}
=
0.
\end{equation}
Since $\mathbf{P}^{\top}\mathbf{P}=\mathbf{I}_{q\times q}$ and $\mathbf{X}^{\top}\mathbf{P} = \mathbf{0}_{p\times q}$, by left multiplying the Equ. (8) by $\mathbf{X}^{\top}$, we have 
\begin{equation}
\mathbf{X}^{\top}\mathbf{B}\mathbf{A}^{\top}
=
\Gamma_{2}^{\top}.
\end{equation}
Put the results back to Equ. (8), we have 
\begin{equation}
\mathbf{B}\mathbf{A}^{\top}
-
\mathbf{P}(\Gamma_{1}+\Gamma_{1}^{\top})
-
\mathbf{X}\mathbf{X}^{\top}\mathbf{B}\mathbf{A}^{\top}
=
0.
\end{equation}
Or equivaliently, 
\begin{equation}
(\mathbf{I}_{n\times n}-\mathbf{X}\mathbf{X}^{\top})\mathbf{B}\mathbf{A}^{\top}
=
\mathbf{P}(\Gamma_{1}+\Gamma_{1}^{\top}).
\end{equation}
Right multiplying Equ. (11) by $\mathbf{P}^{\top}$, we have
\begin{equation}
(\mathbf{I}_{n\times n}-\mathbf{X}\mathbf{X}^{\top})\mathbf{B}\mathbf{A}^{\top}\mathbf{P}^{\top}
=
\mathbf{P}(\Gamma_{1}+\Gamma_{1}^{\top})\mathbf{P}^{\top}
.
\end{equation}
This shows that $(\mathbf{I}_{n\times n}-\mathbf{X}\mathbf{X}^{\top})\mathbf{B}\mathbf{A}^{\top}\mathbf{P}^{\top}$ is a symmertric matrix of order $n\times n$. Then we perfrom economy (or reduced) singular value decomposition (SVD) on $(\mathbf{I}_{n\times n}-\mathbf{X}\mathbf{X}^{\top})\mathbf{B}\mathbf{A}^{\top}$ and get 
$(\mathbf{I}_{n\times n}-\mathbf{X}\mathbf{X}^{\top})\mathbf{B}\mathbf{A}^{\top}=\mathbf{U}\Sigma\mathbf{V}^{\top}$.
Since $(\mathbf{I}_{n\times n}-\mathbf{X}\mathbf{X}^{\top})\mathbf{B}\mathbf{A}^{\top}\mathbf{P}^{\top}$ is symmertric, we have
\begin{equation}
(\mathbf{I}_{n\times n}-\mathbf{X}\mathbf{X}^{\top})\mathbf{B}\mathbf{A}^{\top}\mathbf{P}^{\top}
=
\mathbf{U}\Sigma\mathbf{V}^{\top}\mathbf{P}^{\top}
=
\mathbf{P}\mathbf{V}\Sigma\mathbf{U}^{\top}
\end{equation}
and hence we have $\mathbf{U}=\mathbf{P}\mathbf{V}$ and equivalently $\mathbf{\hat{P}}=\mathbf{U}\mathbf{V}^{\top}$. Note that we can also employ the property of symmertric matrix that every symmertric matrix can be diagonalized to obtain this results. The necessary condition is proofed. 

Now we proof the sufficiency condition. If $\mathbf{\hat{P}}=\mathbf{U}\mathbf{V}^{\top}$, then $\mathbf{\hat{P}}$ satisfies that $\mathbf{\hat{P}}^{\top}\mathbf{\hat{P}}=\mathbf{I}_{q\times q}$ and $\mathbf{X}^{\top}\mathbf{\hat{P}}=\mathbf{0}_{p\times q}$. The first is obvious and now we consider the second one. From the Equ. (4), since $\mathbf{X}^{\top}\mathbf{X}=\mathbf{I}_{p\times p}$, we have  
\begin{equation}
\mathbf{X}^{\top}(\mathbf{I}_{n\times n}-\mathbf{X}\mathbf{X}^{\top})\mathbf{B}\mathbf{A}^{\top}=\mathbf{X}^{\top}\mathbf{B}\mathbf{A}^{\top}-\mathbf{X}^{\top}\mathbf{X}\mathbf{X}^{\top}\mathbf{B}\mathbf{A}^{\top}
=
\mathbf{0}_{p\times n}
.
\end{equation}
It means that $\mathbf{X}^{\top}\mathbf{U}\Sigma\mathbf{V}^{\top}=\mathbf{0}_{p\times p}$ and hence $\mathbf{X}^{\top}\mathbf{U}=\mathbf{0}_{p\times p}$. Then $\mathbf{X}^{\top}\mathbf{\hat{P}}=\mathbf{X}^{\top}\mathbf{U}\mathbf{V}^{\top}=\mathbf{0}_{p\times q}$.

Besides, since
\begin{equation}
\begin{split}
\|\mathbf{B}-\mathbf{P}\mathbf{A}\|_{F}^{2}
=\|\mathbf{B}\|_{F}^{2}+\|\mathbf{P}\mathbf{A}\|_{F}^{2}-2\text{Tr}(\mathbf{B}^{\top}\mathbf{P}\mathbf{A}),
\end{split}
\end{equation}
Until now, if we want to proof that $\mathbf{\hat{P}}=\mathbf{U}\mathbf{V}^{\top}$ is the solution of problem (3), $\text{Tr}(\mathbf{B}^{\top}\mathbf{\hat{P}}\mathbf{A})$ has to be a maximum if $\|\mathbf{B}-\mathbf{\hat{P}}\mathbf{A}\|_{F}^{2}$ is to be a minimum, over all $\mathbf{P}$ satisfying the subject condition in Equ. (3).
Note that by cyclic perturbation which retains the trace unchanged and due to $\mathbf{X}^{\top}\mathbf{\hat{P}}=\mathbf{0}_{p\times q}$, we have 
\begin{equation}
\begin{split}
\text{Tr}(\mathbf{B}^{\top}\mathbf{\hat{P}}\mathbf{A})
&
=
\text{Tr}(\mathbf{B}\mathbf{A}^{\top}\mathbf{\hat{P}}^{\top})
\\
&
=
\text{Tr}((\mathbf{I}_{n\times n}-\mathbf{X}\mathbf{X}^{\top})\mathbf{B}\mathbf{A}^{\top}\mathbf{\hat{P}}^{\top})
\\
&
=
\text{Tr}(\mathbf{U}\Sigma\mathbf{V}^{\top}\mathbf{V}\mathbf{U}^{\top})
\\
&
=
\text{Tr}(\Sigma).
\end{split}
\end{equation}
Now we need to proof that $\text{Tr}(\Sigma)\ge\text{Tr}(\mathbf{B}^{\top}\mathbf{P}\mathbf{A})$ for every $\mathbf{P}$ satisfying that $\mathbf{P}^{\top}\mathbf{P} = \mathbf{I}_{q\times q}, \mathbf{X}^{\top}\mathbf{P} = \mathbf{0}_{p\times q}$. 
Since $\text{Tr}(\mathbf{B}^{\top}\mathbf{P}\mathbf{A})\le$

Since $\mathbf{P}$ is independent of $\text{Tr}(\Sigma)$, for which to be maximum, we need all the diagonal elements in $\Sigma$ to be nonnegative. Hence, once the $\mathbf{V}$ is given, the orientation of $\mathbf{U}$ must be chosen to retain the nonnegative diagonal elements in $\text{Tr}(\Sigma)$. 



For $\text{Tr}(\mathbf{A}\mathbf{B}^{\top}\mathbf{P})$

Note that if the partially known prior were not present, the solution is clearly the solution of the original orthogonal Procrustes problem, i.e., $\mathbf{\hat{P}} = \mathbf{U}\mathbf{V}^{\top}$, where $\mathbf{U}$ and $\mathbf{V}$ are the orthogonal matrices obtained by perfroming economy (aka. reduced) SVD:
$\mathbf{B}\mathbf{A}^{\top} = \mathbf{U}\mathbf{\Sigma}\mathbf{V}^{\top}$. 
The difference between the solutions of the original orthogonal Procrustes problem and its partially known prior version quantify the effect on the residual of requiring $\mathbf{P}$ to be orgothonal to the external prior $\mathbf{P}^{\top}\mathbf{X}$.


In this section, we examine the sensitivity of the solution to perturbation in the data. To measure this sensitivity, we give the relative residuals and the Fro-norm condition numbers of the solutions. The condition number of the matrix $\mathbf{A}$ is defined as $k_{F}(\mathbf{A})=\frac{\sigma_{1}}{\sigma_{r}}$, where $r=\text{rank}(\mathbf{A})$. 


\subsection{Unidimensionality from the Weak LI Conditional Covariance Perspective}


\subsection{Foundational Issues Facilitated by Infinite Test Length Unidimensional MLI1 Modeling}

\subsection{Interpreting Conditional Covariances Geometrically\break to Assess Latent Multidimensional Structure}


\subsection{NIRT-Based Statistical Procedures, Emphasizing Conditional Covariances}


\begin{figure}[h]
%\centerline{\includegraphics{figure03.eps}}
\caption{Projection of item discrimination vectors onto $V_{\theta_T}$ hyperplance for a six item three-dimensional approximate sample structure.}
\end{figure}



\section{Test Fairness}



\subsection{Multidimensional Model for DIF (MMD)}



\subsection{Model-Based Parameterization of the amount of DIF in Various Settings}



\subsection{MMD- Inspired DIF Statistical Procedures}



\begin{figure}[h]
%\centerline{\includegraphics{figure04.eps}}
\caption{Comparison of $\Theta_F$ and $\Theta_R $ distribution with $\Theta_F \vert X_V = k$ and $\Theta_R \vert X_V = k$ distributions.}
\end{figure}

\subsection{Implementation of DIF/DBF Procedures}


\begin{figure}[h]
%\centerline{\includegraphics{figure05.eps}}
\caption{Item discrimination vectors of a 22 item validity sector.}
\end{figure}



\begin{figure}[h]
%\centerline{\includegraphics{figure06.eps}}
\caption{Panel index versus bundle DBF $\hat {\beta}$/item.}
\end{figure}



\section{Formative Assessment Skills Diagnosis: A New Test Paradigm}



\subsection{A Brief Survey of Psychometric Skills Diagnostic Models}



\begin{figure}[h]
%\centerline{\includegraphics[width=254pt]{/figure07.eps}}
\caption{North Carolina End-of-Grade Math Skills Test Subscores.}
\end{figure}



\begin{figure}[h]
%\centerline{\includegraphics{figs/fig08.eps}}
\caption{PSAT Score Report \textit{Plus} Skills Mastery Reporting.}
\end{figure}




\subsection{The Unified Model and Generalizations Making it Useful}



\subsection{Application of the Unified Model to PSAT Data}


\subsection{Skills Diagnosis: The New Paradigm?}


\section{Dimensionality, Equity, and Diagnostic Software}


\section{Concluding Remarks}

\vspace{\fill}\clearpage

\begin{thebibliography}

\bibitem Ackerman, T.A. (1992). A didactic explanation of item bias, item impact, and item validity from a multidimensional perspective. \textit{Journal of Educational Measurement}, \textit{29}, 67--91.

\bibitem Angoff, W.H. (1993). Perspectives on differential item functioning methodology. In P.W. Holland \& H. Wainer (Eds.), \textit{Differential item functioning}(pp.~3--24). Hillsdale, NJ: Lawrence Erlbaum Associates.

\bibitem Bolt, D., Froelich, A.G., Habing, B., Hartz, S., Roussos, L., \& Stout, W. (in press). \textit{An applied and foundational research project addressing DIF, impact, and equity: With applications to ETS test development} (ETS Technical Report). Princeton, NJ: ETS.

Chang, H., Mazzeo, J., \& Roussos, L. (1996). Detecting DIF for polytomously scored items: an adaptation of the SIBTEST procedure. \textit{Journal of Educational Measurement}, \textit{33}, 333--353

\bibitem Chang, H., \& Stout, W. (1993). The asymptotic posterior normality of the latent trait in an IRT model. \textit{Psychometrika}, \textit{58}, 37--52.


\bibitem DiBello, L., Stout, W., \& Roussos, L. (1995). Unified cognitive/psychometric diagnostic assessment likelihood-based classification techniques. In P. Nichols, S. Chipman, \& R. Brennen (Eds.), \textit{Cognitively diagnostic assessment} (pp.~361--389). Hillsdale, NJ: Earlbaum.

\bibitem Doignon, J.-P., \& Falmagne, J.-C. (in press). \textit{Knowledge spaces}. Berlin Springer-Verlag.

\bibitem Dorans, N.J., \& Kulick, E. (1986). Demonstrating the utility of the standardization approach to assessing unexpected differential item performance on the Scholastic Aptitude Test. \textit{Journal of Educational Measurement}, \textit{23}, 355--368.

\bibitem Douglas, J. (1997). Joint consistency of nonparametric item characteristic curve and ability estimation. \textit{Psychometrika}, \textit{62}, 7--28.

\bibitem Douglas, J.A. (2001). Asymptotic identifiability of nonparametric item response models. \textit{Psychometrika}, \textit{66}, 531--540.

\bibitem Douglas J.A., \& Cohen A. (2001). Nonparametric ICC estimation to assess fit of parametric models. \textit{Applied Psychological Measurement}, \textit{25}, 234--243.

\bibitem Douglas, J., Kim, H.R., Habing, B., \& Gao, F. (1998) Investigating local dependence with conditional covariance functions. \textit{Journal of Educational and Behavioral Statistics}, \textit{23}, 129--151.

\bibitem Douglas, J., Roussos, L., \& Stout, W., (1996). Item bundle DIF hypothesis testing: Identifying suspect bundles and assessing their DIF. \textit{Journal of Educational Measurement}, \textit{33}, 465--484.

\bibitem Douglas, J., Stout, W., \& DiBello, L. (1996). A kernel smoothed version of SIBTEST with applications to local DIF inference and unction estimation. \textit{Journal of Educational and Behavioral Statistics}, \textit{21}, 333--363.

\bibitem Ellis, J.L., \& Junker, B.W. (1997). Tail-measurability in monotone latent variable models. \textit{Psychometrika}, \textit{62}, 495--524.

\bibitem Embretson (Whitely), S.E. (1980). Multicomponent latent trait models for ability tests \textit{Psychometrika}, \textit{45}, 479--494.

\bibitem Embretson, S.E. (1984). A general latent trait model for response processes. \textit{Psychometrika}, \textit{49}, 175--186.

\bibitem Embretson, S. E. (Ed.). (1985), \textit{Test design: Developments in psychology and psychometrics} (pp.~195--218, chap.~7). Orlando, FL: Academic Press.

\bibitem Fischer, G.H. (1973). The linear logistic test model as an instrument in educational research. \textit{Acta Psychologica}, \textit{37}, 359--374.

\bibitem Froelich, A.G., \& Habing, B. (2002, July). A study of methods for selecting the AT subtest in the DIMTEST procedure. Paper presented at the 2002 Annual Meeting of the Psychometrika Society, University of North Carolina at Chapel Hill.

\bibitem Gierl, M.J., Bisanz, J., Bisanz, G., Boughton, K., \& Khaliq, S. (2001). Illustrating the utility of differential bundle functioning analyses to identify and interpret group differences on achievement tests. \textit{Educational Measurement: Issues and Practice}, \textit{20}, 26--36.

\bibitem Gierl, M.J., \& Khaliq, S.N. (2001). Identifying sources of differential item and bundle functioning on translated achievement tests. \textit{Journal of Educational Measurement}, \textit{38}, 164--187.

\bibitem Gierl, M.J., Bisanz, J., Bisanz, G.L., \& Boughton, K.A. (2002, April). Identifying content and cognitive skills that produce gender differences in mathematics: A demonstration of the DIF analysis framework. Paper presented at the annual meeting of the National Council on Measurement in Education, New Orleans, LA.

\bibitem Haberman, S.J (1977). Maximum likelihood estimates in exponential response models. \textit{The Annals of Statistics}, \textit{5}, 815--841.

\bibitem Habing, B. (2001). Nonparametric regression and the parametric bootstrap for local dependence assessment. \textit{Applied Psychological Measurement}, \textit{25}, 221--233.

\bibitem Haertel, E. (1989). Using restricted latent class models to map the skill structure of achievement items. \textit{Journal of Educational Measurement}, \textit{26}, 301--321.

\bibitem Hartz, S.M. (2002). \textit{A Bayesian framework for the Unified Model for assessing cognitive abilities: blending theory with practicality}. Unpublished doctoral dissertation, University of Illinois, Urbana-Champaign, Department of Statistics.

\bibitem Holland, P.W. (1990a). On the sampling theory foundations of item response theory models. \textit{Psychometrika}, \textit{55}, 577--601.

\bibitem Holland, P.W. (1990b). The Dutch identity: a new tool for the study of item response models. \textit{Psychometrika}, \textit{55}, 5--18.

\bibitem Holland, P.W., \& Rosenbaum, P.R. (1986). Conditional association and unidimensionality in monotone latent variable models. \textit{The Annals of Statistics}, \textit{14}, 1523--1543.

\bibitem Holland, W.P., \& Thayer, D.T. (1988). Differential item performance and the Mantel-Haenszel procedure. In H. Wainer \& H.I. Braun (Eds.), \textit{Test validity} (pp.~129--145). Hillsdale, NJ: Lawrence Earlbaum Associates.

\bibitem Jiang, H., \& Stout, W. (1998). Improved Type I error control and reduced estimation bias for DIF detection using SIBTEST. \textit{Journal of Educational and Behavioral Statistics}, \textit{23}, 291--322.

\bibitem Junker, B.W. (1993). Conditional association, essential independence, and monotone unidimensional latent variable models. \textit{Annals of Statistics}, \textit{21}, 1359--1378.

\bibitem Junker, B.W. (1999). \textit{Some statistical models and computational methods that may be useful for cognitively-relevant assessment}. Prepared for the National Research Council Committee on the Foundations of Assessment. Retrieved April 2, 2001, from \mbox{http://www.stat.cmu.edu/$\sim $brian/nrc/cfa/}

\bibitem Junker, B.W., \& Ellis, J.L. (1998). A characterization of monotone unidimensional latent variable models. \textit{Annals of Statistics}, \textit{25}(3), 1327--1343.

\bibitem Junker, B. W. \& Sijtsma, K. (2001). Nonparametric item response theory in action: an overview of the special issue. \textit{Applied Psychological Measurement}, \textit{25}, 211--220.

\bibitem Koedinger, K.R., \& MacLaren, B.A. (2002). Developing a pedagogical domain theory of early algebra problem solving (CMU-HCII Tech. Rep.~02--100). Pittsburgh, PA: Carnegie Mellon University, School of Computer Science.

\bibitem Li, H. \& Stout, W. (1996). A new procedure for detecting crossing DIF. \textit{Psychometrika}, \textit{61}, 647--677.

\bibitem Kok, F. (1988). Item bias and test multidimensionality. In R. Langeheine \& J. Rost (Eds.), \textit{Latent trait and latent models} (pp.~263--275). New York, NY: Plenum Press.

\bibitem Linn, R.L. (1993). The use of differential item functioning statistics: A discussion of current practice and future implications. In P.W. Holland \& H. Wainer (Eds.), \textit{Differential item functioning} (pp.~349--364). Hillsdale, NJ: Lawrence Erlbaum Associates.

%\newpage

\bibitem Lord, F.M. (1980) \textit{Applications of item response theory to practical testing problems}. Lawrence Erlbaum Associates, Hinsdale, NJ.

\bibitem McDonald, R.P. (1994). Testing for approximate dimensionality. In D. Laveault, B.D. Zumbo, M.E. Gessaroli, \& M.W. Boss (Eds.), \textit{Modern theories of measurement: Problems and issues} (pp.~63--86). Ottawa, Canada: University of Ottawa.

\bibitem Maris, E. (1995). Psychometric latent response models. \textit{Psychometrika}, \textit{60}, 523--547.

\bibitem Mislevy, R.J. (1994). Evidence and inference in educational assessment. \textit{Psychometrika}, \textit{59}, 439--483.

\bibitem Mislevy, R.J. Almond, R.G., Yan, D., \& Steinberg, L.S. (1999). Bayes nets in educational assessment: Where do the numbers come from? In K.B. Laskey \& H. Prade (Eds.), \textit{Proceedings of the Fifteenth Conference on Uncertainty in Artificial Intelligence} (pp.~437--446). San Francisco, CA: Morgan Kaufmann.

\bibitem Mislevy, R., Steinberg, L. \& Almond, R. (in press). On the structure of educational assessments. \textit{Measurement: Interdisciplinary research and perspective}. Hillsdale, NJ: Lawrence Erlbaum Associates.

\bibitem Mokken, R.J. (1971). \textit{A theory and procedure of scale analysis}. The Hague: Mouton.

\bibitem Molenaar, I.W., \& Sijtsma, K. (2000). \textit{User's manual MSP5 for Windows: A program for Mokken Scale Analysis for Polytomous Items. Version 5.0} [Software manual]. Groningen: ProGAMMA.

\bibitem Nandakumar, R. (1993). Simultaneous DIF amplification and cancellation: Shealy-Stout's test for DIF. \textit{Journal of Educational Measurement}, \textit{30}, 293--311.

\bibitem Nandakumar, R., \& Roussos, L. (in press). Evaluation of CATSIB procedure in pretest setting. \textit{Journal of Educational and Behavioral Statistics}.

\bibitem Nandakumar, R., \& Stout, W. (1993). Refinements of Stout's procedure for assessing latent trait unidimensionality. \textit{Journal of Educational Statistics}, \textit{18}, 41--68.

\bibitem O'Neill, K.A., \& McPeek, W.M. (1993). Item and test characteristics that are associated with differential item functioning. In P.W. Holland \& H. Wainer (Eds.), \textit{Differential item functioning} (pp.~255--276). Hillsdale, NJ: Lawrence Erlbaum Associates.

\bibitem Pellegrino, J.W., Chudowski, N., \& Glaser, R (Eds.). (2001). \textit{Knowing what students know: The science and design of educational assessment} (chap.~4, pp.~111--172) Washington, DC: National Academy Press.

\bibitem Philipp, W. \& Stout, W. (1975). \textit{Almost sure convergence principles for sums of dependent random variables} (American Mathematical Society Memoir No. 161). Providence, RI: American Mathematical Society.

\bibitem Ramsay, J.O. (2000). TESTGRAF: \textit{A program for the graphical analysis of multiple choice test and questionnaire data} (TESTGRAF user's guide for TESTGRAF98 software). Montreal, Quebec: Author. Versions available for Windows\textregistered, DOS, and Unix. The Windows\textregistered\ version was retreived November 11, 2002 from \mbox{ftp://ego.psych.mcgill.ca/pub/ramsay/testgraf/TestGraf98.wpd}

\bibitem Ramsey, P.A. (1993). Sensitivity review: the ETS experience as a case study. In P.W. Holland \& H. Wainer (Eds.), \textit{Differential item functioning} (pp.~367--388). Hillsdale, NJ: Lawrence Erlbaum Associates.

\bibitem Rossi, N., Wang, W. \& Ramsay, J.O. (in press). Nonparametric item response function estimates with the EM algorithm. \textit{Journal of Educational and Behavioral Statistics}.

\bibitem Roussos, L., \& Stout, W. (1996a). DIF from the multidimensional perspective. \textit{Applied Psychological Measurement}, \textit{20}, 335--371.

\bibitem Roussos, L., \& Stout, W. (1996b). Simulation studies of the effects of small sample size and studied item parameters on SIBTEST and Mantel-Haenszel Type 1 error performance. \textit{Journal of Education Measurement}, \textit{33}, 215--230.

\bibitem Roussos, L.A., Stout, W.F., \& Marden, J. (1998). Using new proximity measures with hierarchical cluster analysis to detect multidimensionality. \textit{Journal of Educational Measurement}, \textit{35}, 1--30.

\bibitem Roussos, L.A., Schnipke, D.A., \& Pashley, P.J. (1999). A generalized formula for the Mantel-Haenszel differential item functioning parameter. \textit{Journal of Educational and Behavioral Statistics}, \textit{24}, 293--322.

\bibitem Shealy, R.T. (1989). \textit{An item response theory-based statistical procedure for detecting concurrent internal bias in ability tests}. Unpublished doctoral dissertation, Department of Statistics, University of Illinois, Urbana-Champaign.

Shealy, R.,\& Stout, W. (1993a). A model-based standardization approach that separates true bias/DIF from group ability differences and detects test bias/DTF as well as item bias/DIF. \textit{Psychometrika}, \textit{58}, 159--194.

\bibitem Shealy, R., \& Stout, W. (1993b). An item response theory model for test bias and differential test functioning. In P. Holland \& H. Wainer (Eds.), \textit{Differential item functioning}(pp.~197--240). Hillsdale, NJ: Lawrence Erlbaum.

\bibitem Sijtsma, K. (1998). Methodology review: nonparametric IRT approaches to the analysis of dichotomous item scores. \textit{Applied Psychological Measurement}, \textit{22}, 3--32.

\bibitem Sternberg, R.J. (1985). \textit{Beyond IQ: A triarchic theory of human intelligence}. New York, NY: Cambridge University Press.

\bibitem Stout, W. (1987). A nonparametric approach for assessing latent trait unidimensionality. \textit{Psychometrika}, \textit{52}, 589--617.

\bibitem Stout, W. (1990). A new item response theory modeling approach with applications to unidimensionality assessment and ability estimation. \textit{Psychometrika}, \textit{55}, 293--325.

\bibitem Stout, W., Froelich, A.G., \& Gao, F. (2001). Using resampling to produce an improved DIMTEST procedure. In A. Boomsma, M.A.J. van Duijn, T.A.B. Snijders (Eds.), \textit{Essays on item response theory} (pp.~357--376). New York, NY: Springer-Verlag.

\bibitem Stout, W., Habing, B., Douglas, J., Kim, H.R., Roussos, L., \& Zhang, J. (1996). Conditional covariance based nonparametric multidimensionality assessment. \textit{Applied Psychological Measurement}, \textit{20}, 331--354.

\bibitem Stout, W., Li, H., Nandakumar, R., \& Bolt, D. (1997). MULTISIB---A procedure to investigate DIF when a test is intentionally multidimensional. \textit{Applied Psychological Measurement}, \textit{21}, 195--213.

\bibitem Suppes, P., \& Zanotti, M. (1981). When are probabilistic explanations possible? \textit{Synthese}, \textit{48}, 191--199.

\bibitem Tatsuoka, K. K. (1990). Toward an integration of item-response theory and cognitive error diagnosis. In N. Frederiksen, R. Glazer, A. Lesgold, \& M.G. Shafto (Eds.), \textit{Diagnostic monitoring of skill and knowledge acquisition} (pp.~453--488). Hillsdale, NJ: Lawrence Erlbaum Associates.

%\newpage

\bibitem Tatsuoka, K. K. (1995). Architecture of knowledge structures and cognitive diagnosis: A statistical pattern recognition and classification approach. In P. Nichols, S. Chipman, \& R. Brennen (Eds.), \textit{Cognitively diagnostic assessment}. Hillsdale, NJ: Earlbaum. 327--359.

\bibitem Thissen, D., \& Wainer, H. (2001). \textit{Test scoring}. Hillsdale, NJ: Lawrence Erlbaum Associates.

\bibitem Trachtenberg, F., \& He, X. (2002). One-step joint maximum likelihood estimation for item response theory models. Submitted for publication.

\bibitem Tucker, L.R., Koopman, R.F., \& Linn, R.L. (1969). Evaluation of factor analytic research procedures by means of simulated correlation matrices. \textit{Psychometrika}, \textit{34}, 421--459.

\bibitem Wainer, H., \& Braun, H.I. (1988). \textit{Test validity}. Hillsdale, NJ: Lawrence Erlbaum Associates. Zhang, J., \& Stout, W. (1999a). Conditional covariance structure of generalized compensatory multidimensional items. \textit{Psychometrika}, 64, 129--152.

\bibitem Whitely, S.E. (1980). (See Embretson, 1980)

\bibitem Zhang, J., \& Stout, W. (1999). The theoretical DETECT index of dimensionality and its application to approximate simple structure. \textit{Psychometrika}, \textit{64}, 213--249.
\end{thebibliography}
\vspace{\fill}

%\vfill\eject
\end{document}
